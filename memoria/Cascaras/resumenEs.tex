%---------------------------------------------------------------------
%
%                      resumen.tex
%
%---------------------------------------------------------------------
%
% Contiene el cap�tulo del resumen en ingl�s.
%
% Se crea como un cap�tulo sin numeraci�n.
%
%---------------------------------------------------------------------

\chapter{Resumen}
\cabeceraEspecial{Resumen}

Actualmente, las redes de sensores inal�mbricas est�n sometidas a impedimentos que dificultan su desarrollo y despliegue. Un ejemplo de estas limitaciones es la creciente saturaci�n del espectro radioel�ctrico en las bandas de frecuencia libres.

\vspace{.1cm}

Las redes cognitivas, apoy�ndose en un modelo de comunicaci�n cooperativo, representan un nuevo paradigma que tiene por objetivo un uso  m�s eficiente del espectro y la mejora de las comunicaciones inal�mbricas. Las redes de sensores cognitivas aplican propiedades cognitivas en redes de sensores comunes, desarrollando as� nuevas estrategias para mitigar la ineficiencia en las comunicaciones que sufren estas redes.

\vspace{.1cm}

Es importante investigar modelos cognitivos para explorar los beneficios que pueden aportar a nuestras redes de sensores, especialmente limitadas en energ�a y recursos. Sin embargo, pocas plataformas y con funcionalidades generalmente pobres o demasiado espec�ficas permiten su estudio debido a un estado del arte a�n inmaduro. La mayor�a de las investigaciones se llevan a cabo sobre simuladores, los cuales muestran resultados parciales o incompletos.

\vspace{.1cm}

Este documento presenta el dise�o y la implementaci�n una plataforma vers�til que integra propiedades cognitivas en las redes de sensores. Se han combinado m�dulos hardware y software en un instrumento que ayude a la investigaci�n de las redes de sensores cognitivas. La plataforma permite comunicaci�n sobre tres bandas distintas de radiofrecuencia, siendo la �nica plataforma para redes de sensores que lo permite, y el hardware se ajusta a exigencias de tama�o, coste y energ�a. Adem�s, su dise�o modular y escalable es ampliamente adaptable a cualquier aplicaci�n para redes de sensores.

\vspace{.5cm}

\begin{table}[h!]
\Large
\scalebox{0.8}{
\begin{tabular}{ l l }
\textbf{\emph{KEY WORDS}}:	& \emph{redes cognitivas}, \emph{radio cognitiva}, \emph{redes de sensores inal�mbricas}, \\ 
 				& \emph{plataforma}, \emph{banco de pruebas}, \emph{nodo}, \emph{dispositivo},\\
				& \emph{redes de sensores inal�mbricas cognitivas}.
\end{tabular}}
\end{table}

\endinput
% Variable local para emacs, para  que encuentre el fichero maestro de
% compilaci�n y funcionen mejor algunas teclas r�pidas de AucTeX
%%%
%%% Local Variables:
%%% mode: latex
%%% TeX-master: "../Tesis.tex"
%%% End:
