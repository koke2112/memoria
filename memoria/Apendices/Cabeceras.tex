%---------------------------------------------------------------------
%
%                          Ap�ndice C
%
%---------------------------------------------------------------------

\chapter{Cabecera y definici�n de funciones}
\label{ap1:cabeceras}

En este ap�ndice se incluyen las cabeceras y declaraciones de las principales funciones implementadas en el sistema.

%\begin{FraseCelebre}
%\begin{Frase}
%...
%\end{Frase}
%\begin{Fuente}
%...
%\end{Fuente}
%\end{FraseCelebre}

%\begin{resumen}
%...
%\end{resumen}

%-------------------------------------------------------------------
\section{extractMessInfo}
%-------------------------------------------------------------------
\label{ap1:extractMessInfo}
\begin{lstlisting}[style=C]
/******************************************************************/
/* Take out the NNID and frameID headers of the message received  */
/******************************************************************/
/* INPUT                                                          */
/*  -message:   Message received                                  */
/*                                                                */
/* OUTPUT                                                         */
/*  -message:   Message received without NNID and frameID fields. */
/*  -nnid:      NNID field                                        */
/*  -frameID:   frameID field                                     */
/*                                                                */
/* RETURNS                                                        */
/*  - ERROR if there is any problem in the process.               */
/*  - OK if the process is succesful.                             */
/*                                                                */
/******************************************************************/
int extractMessInfo (char *nnid, char *frameID, char *message);
\end{lstlisting}


%-------------------------------------------------------------------
\section{checkAlarm}
%-------------------------------------------------------------------
\label{ap1:checkAlarm}
\begin{lstlisting}[style=C]
/******************************************************************/
/* Check if there are new alarms in the alarm.dat file.           */
/******************************************************************/
/* INPUT                                                          */
/*  -lastAlarm: The alarmID of the last alarm used.               */
/*                                                                */
/* RETURNS                                                        */
/*  -ERROR if there is not more alarms now.                       */
/*  -The index of the line in the file with the next alarm to send*/
/******************************************************************/
int checkAlarm (unsigned int lastAlarm);
\end{lstlisting}

%-------------------------------------------------------------------
\section{sendLocalAlarm}
%-------------------------------------------------------------------
\label{ap1:sendLocalAlarm}
\begin{lstlisting}[style=C]
/******************************************************************/
/* Read an alarm in the 'alarmIndex' line from the 'alarm.dat'    */
/* file and send it.                                              */
/******************************************************************/
/* INPUT                                                          */
/*  -dnid:       DNID field                                       */
/*  -nnid:       NNID field                                       */
/*  -nid:        SNID field                                       */
/*  -alarmIndex: The alarm index of the alarm to read.            */
/*                                                                */
/* RETURNS                                                        */
/*  -The alarmID of the alarm sended.                             */
/*  -0 if there is any problem in the process.                    */
/******************************************************************/
unsigned int sendLocalAlarm (char *dnid, char *nnid, char *nid, int alarmIndex);
\end{lstlisting}

%-------------------------------------------------------------------
\section{exeDATA}
%-------------------------------------------------------------------
\label{ap1:exeDATA}
\begin{lstlisting}[style=C]
/******************************************************************/
/* Execute the DATA reception process.                            */
/******************************************************************/
/* INPUT                                                          */
/*  -nid:         NID of the own node.                            */
/*  -DATAmessage: DATA message received.                          */
/*  -lt_rte:      Lifetime of a routing table entry.              */
/*                                                                */
/* OUTPUT                                                         */
/*  -auxdnid:     DNID of the ACK, when a route has to be         */
/*                rediscovered.                                   */
/*                                                                */
/* RETURNS                                                        */
/*  -ERROR if there is any problem in the process.                */
/*  -OK if the process is succesful.                              */
/******************************************************************/
int exeDATA(char *nid, char *DATAmessage, int lt_rte, char *auxdnid);
\end{lstlisting}

%-------------------------------------------------------------------
\section{sendDATA}
%-------------------------------------------------------------------
\label{ap1:sendDATA}
\begin{lstlisting}[style=C]
/******************************************************************/
/* Send a unicast DATA message.                                   */
/******************************************************************/
/* INPUT                                                          */
/*  -nnid:        Next node ID in the route.                      */
/*  -DATAmessage: The part of the message with the info.          */
/******************************************************************/
void sendDATA (char *nnid, char *DATAmessage);
\end{lstlisting}

%-------------------------------------------------------------------
\section{exeACK}
%-------------------------------------------------------------------
\label{ap1:exeACK}
\begin{lstlisting}[style=C]
/******************************************************************/
/* Execute the ACK reception process.                             */
/******************************************************************/
/* INPUT                                                          */
/*  -nid            NID of the own node.                          */
/*  -ACKmessage:    ACK message received.                         */
/*  -lt_rte:        Lifetime of a routing table entry.            */
/* RETURNS                                                        */
/*  -alarmID when the ACK messages corresponds with a local       */
/*   alarm sended.                                                */
/*  -0 in other cases.                                            */
/******************************************************************/
unsigned int exeACK (char *nid, char *ACKmessage, int lt_rte);
\end{lstlisting}

%-------------------------------------------------------------------
\section{sendACK}
%-------------------------------------------------------------------
\label{ap1:sendACK}
\begin{lstlisting}[style=C]
/******************************************************************/
/* Send a unicast ACK message.                                    */
/******************************************************************/
/* INPUT                                                          */
/*  -nnid           Next node ID in the route.                    */
/*  -ACKmessage:    The part of the message with the info.        */
/******************************************************************/
void sendACK (char *nnid, char *ACKmessage);
\end{lstlisting}

%-------------------------------------------------------------------
\section{serialData2queue}
%-------------------------------------------------------------------
\label{ap1:serialData2queue}
\begin{lstlisting}[style=C]
/******************************************************************/
/* Read the serial data and put the messages received between     */
/* brackets ([Hello world!]) in the message queue.		  */
/******************************************************************/
/* INPUT                                                          */
/*  -serialData: Data readed in the serial port.                  */
/*  -Q:          Message queue.			                  */
/******************************************************************/
void serialData2queue (char *serialData, queue Q);
\end{lstlisting}

%-------------------------------------------------------------------
\section{initSeqNum}
%-------------------------------------------------------------------
\label{ap1:initSeqNum}
\begin{lstlisting}[style=C]
/******************************************************************/
/* Inicialize the sequence number with time(NULL) for making a    */
/* strong system against resets.				  */
/******************************************************************/
void initSeqNum (void);
\end{lstlisting}

%-------------------------------------------------------------------
\section{updateSeqNum}
%-------------------------------------------------------------------
\label{ap1:updateSeqNum}
\begin{lstlisting}[style=C]
/******************************************************************/
/* Update the sequence number.					  */
/******************************************************************/
/* INPUT							  */
/*  -seqN: The sequence number received in a RREQ or PREQ messages*/
/*  								  */
/* RETURNS							  */
/*  -The sequence number updated.				  */
/******************************************************************/
unsigned long int updateSeqNum (unsigned long int seqN);
\end{lstlisting}

%-------------------------------------------------------------------
\section{checkRREQ}
%-------------------------------------------------------------------
\label{ap1:checkRREQ}
\begin{lstlisting}[style=C]
/******************************************************************/
/* Check the RREQ table to verify that the RREQ is updated.	  */
/******************************************************************/
/* INPUT							  */
/*  -snid: Source node id of the RREQ message.			  */
/*  -seqN: The sequence number received in a RREQ message.	  */
/*  								  */
/* RETURNS							  */
/*  -OK to add the new entry.					  */
/*  -ERROR to discard the new entry.				  */
/******************************************************************/
int checkRREQ (char *snid, unsigned long int seqN);
\end{lstlisting}

%-------------------------------------------------------------------
\section{exeRREQ}
%-------------------------------------------------------------------
\label{ap1:exeRREQ}
\begin{lstlisting}[style=C]
/******************************************************************/
/* Execute the RREQ process.					  */
/******************************************************************/
/* INPUT							  */
/*  -nid:  	  NID of the own node.  			  */
/*  -max_hops:    Maximum number of hops setted in the node.conf  */
/*  -RREQmessage: RREQ message received.			  */
/*  -lt_rte: 	  Lifetime of a routing table entry.		  */
/*  								  */
/* RETURNS							  */
/*  -OK if the process is succesful.				  */
/*  -ERROR if there is any problem in the process.		  */
/*                                                                */
/******************************************************************/
int exeRREQ (char *nid, int max_hops, char *RREQmessage, int lt_rte);
\end{lstlisting}

%-------------------------------------------------------------------
\section{sendRREQ}
%-------------------------------------------------------------------
\label{ap1:sendRREQ}
\begin{lstlisting}[style=C]
/******************************************************************/
/* Send a broadcast RREQ message. If the sequence number in the   */
/* argument is '0', the RREQ is local and the function increase   */
/* the local seqNum. If the argument is different of 0, the RREQ  */
/* is not local, it is a rebroadcast and it have its own sequence */
/* number.							  */
/******************************************************************/
/* INPUT							  */
/*  -snid: SNID field.			  			  */
/*  -seqN: The sequence number of the RREQ, or 0 if it is a local */
/*         RREQ which has to be sent.				  */
/*  -dnid: DNID field.						  */
/*  -pnid: PNID field.						  */
/*  -hops: The number of hops of the frame.			  */
/*  								  */
/* RETURNS							  */
/*  -OK when make a RREQ rebroadcast. 				  */
/*  -Sequence number when sends a local RREQ.			  */
/******************************************************************/
unsigned long int sendRREQ (char *snid, unsigned long int seqN, char *dnid, char *pnid, int hops);
\end{lstlisting}

%-------------------------------------------------------------------
\section{checkPREQ}
%-------------------------------------------------------------------
\label{ap1:checkPREQ}
\begin{lstlisting}[style=C]
/******************************************************************/
/* Check the PREQ table to verify that the PREQ is updated.	  */
/******************************************************************/
/* INPUT							  */
/*  -snid: Source node id of the PREQ message.			  */
/*  -seqN: The sequence number received in a PREQ message.	  */
/*  								  */
/* RETURNS							  */
/*  -OK to add the new entry.					  */
/*  -ERROR to discard the new entry.				  */
/******************************************************************/
int checkPREQ (char *snid, unsigned long int seqN);
\end{lstlisting}

%-------------------------------------------------------------------
\section{exePREQ}
%-------------------------------------------------------------------
\label{ap1:exePREQ}
\begin{lstlisting}[style=C]
/******************************************************************/
/* Execute the PREQ process.					  */
/******************************************************************/
/* INPUT							  */
/*  -nid:  	  NID of the own node.  			  */
/*  -PREQmessage: PREQ message received.			  */
/*  -lt_rte: 	  Lifetime of a routing table entry.		  */
/*  								  */
/* RETURNS							  */
/*  -OK if the process is succesful.				  */
/*  -ERROR if there is any problem in the process.		  */
/******************************************************************/
int exePREQ (char *nid, char *PREQmessage, int lt_rte);
\end{lstlisting}

%-------------------------------------------------------------------
\section{sendPREQ}
%-------------------------------------------------------------------
\label{ap1:sendPREQ}
\begin{lstlisting}[style=C]
/******************************************************************/
/* Send a broadcast PREQ message. If the sequence number in the   */
/* argument is '0', the PREQ is local and the function increase   */
/* the local seqNum. If the argument is different of 0, the PREQ  */
/* is not local, it is a rebroadcast and it have its own sequence */
/* number.							  */
/******************************************************************/
/* INPUT							  */
/*  -snid: SNID field.			  			  */
/*  -seqN: The sequence number of the PREQ, or 0 if it is a local */
/*         PREQ which has to be sent.				  */
/*  -pnid: PNID field.						  */
/*  -hops: The number of hops of the frame.			  */
/*  								  */
/* RETURNS							  */
/*  -OK when make a PREQ rebroadcast. 				  */
/*  -Sequence number when sends a local PREQ.			  */
/******************************************************************/
unsigned long int sendPREQ (char *snid, unsigned long int seqN, char *pnid, int hops);
\end{lstlisting}

%-------------------------------------------------------------------
\section{exeRREP}
%-------------------------------------------------------------------
\label{ap1:exeRREP}
\begin{lstlisting}[style=C]
/******************************************************************/
/* Execute the RREP process.					  */
/******************************************************************/
/* INPUT							  */
/*  -nid:  	  NID of the own node.  			  */
/*  -max_hops:    Maximum number of hops setted in the node.conf  */
/*  -RREPmessage: RREP message received.			  */
/*  -lt_rte: 	  Lifetime of a routing table entry.		  */
/*  								  */
/* RETURNS							  */
/*  -The sequence number (seqN > 0) when the RREP corresponds	  */
/*   with one local RREQ.					  */
/*  -0 in the other cases.					  */
/******************************************************************/
unsigned long int exeRREP (char *nid, int max_hops, char *RREPmessage, int lt_rte);
\end{lstlisting}

%-------------------------------------------------------------------
\section{sendRREP}
%-------------------------------------------------------------------
\label{ap1:sendRREP}
\begin{lstlisting}[style=C]
/******************************************************************/
/* Send a unicast RREP message.					  */
/******************************************************************/
/* INPUT							  */
/*  -nnid: NNID field.			  			  */
/*  -pnid: PNID field.						  */
/*  -snid: SNID field.						  */
/*  -seqN: The sequence number.					  */
/*  -hops: The number of hops of the frame.			  */
/*  -dnid: DNID field.			  			  */
/******************************************************************/
void sendRREP (char *nnid, char *pnid, char *snid, unsigned long int seqN, int hops, char *dnid);
\end{lstlisting}

%-------------------------------------------------------------------
\section{exePREP}
%-------------------------------------------------------------------
\label{ap1:exePREP}
\begin{lstlisting}[style=C]
/******************************************************************/
/* Execute the PREP process.					  */
/******************************************************************/
/* INPUT							  */
/*  -nid:  	  NID of the own node.  			  */
/*  -PREPmessage: PREP message received.			  */
/*  -lt_rte: 	  Lifetime of a routing table entry.		  */
/* RETURNS							  */
/*  -OK if the process is succesful.				  */
/*  -ERROR if there is any problem in the process.		  */
/******************************************************************/
int exePREP (char *nid, char *PREPmessage, int lt_rte);
\end{lstlisting}

%-------------------------------------------------------------------
\section{sendPREP}
%-------------------------------------------------------------------
\label{ap1:sendPREP}
\begin{lstlisting}[style=C]
/******************************************************************/
/* Send a unicast PREP message.					  */
/******************************************************************/
/* INPUT							  */
/*  -nnid: 	NNID field.			  		  */
/*  -dnid:  	DNID field.			  		  */
/*  -timeStamp: Timestamp in the PREP received, or 0 if it is a   */
/*              local PREP message.				  */
/*  -snid: 	SNID field.					  */
/*  -latitude: 	Latitude in the PREP received, or NULL if it is a */
/*              local PREP message.				  */
/*  -longitude: Longitude in the PREP received, or NULL if it is a*/
/*              local PREP message.				  */
/******************************************************************/
void sendPREP (char *nnid, char *dnid, unsigned long int timeStamp, char *snid, char *latitude, char *longitude);
\end{lstlisting}


%-------------------------------------------------------------------
\section{checkNewRoute}
%-------------------------------------------------------------------
\label{ap1:checkNewRoute}
\begin{lstlisting}[style=C]
/******************************************************************/
/* Check if the new route is already in the routing table and	  */
/* also check the update conditions.				  */
/******************************************************************/
/* INPUT                                                          */
/*  -dnid:   Destination node of the route.			  */
/*  -seqn:   Sequence number of the route.			  */
/*  -hops:   The number of hops of the route till the own node.   */
/*  -lt_tre: Lifetime of a routing table entry.			  */
/* RETURNS                                                        */
/*  - OK when the routing table is OK to be updated.              */
/*  - ERROR if the routing table is already updated.              */
/******************************************************************/
int checkNewRoute (char *dnid, unsigned long int seqn, int hops, int lt_rte);
\end{lstlisting}

%-------------------------------------------------------------------
\section{newRoute}
%-------------------------------------------------------------------
\label{ap1:newRoute}
\begin{lstlisting}[style=C]
/******************************************************************/
/* Add a new route to the routing table.			  */
/******************************************************************/
/* INPUT                                                          */
/*  -dnid: Destination node of the route.			  */
/*  -nnid: Next node in the route to the DNID.			  */
/*  -seqn: Sequence number of the route.			  */
/*  -hops: The number of hops of the route till the own node.     */
/* RETURNS                                                        */
/*  -OK if the process is succesful.				  */
/*  -ERROR if there is any problem in the process.		  */
/******************************************************************/
int newRoute(char *dnid,char *nnid,unsigned long int seqn,int hops);
\end{lstlisting}

%-------------------------------------------------------------------
\section{getRoute}
%-------------------------------------------------------------------
\label{ap1:getRoute}
\begin{lstlisting}[style=C]
/******************************************************************/
/* Look for a valid route to the dnid.				  */
/******************************************************************/
/* INPUT                                                          */
/*  -dnid:   Destination node of the route.			  */
/*  -lt_rte: Lifetime of a routing table entry.			  */
/* OUTPUT							  */
/*  -nnid:   Next node in the route to the DNID.		  */
/* RETURNS                                                        */
/*  -OK if the process is succesful.				  */
/*  -ERROR if there is any problem in the process.		  */
/******************************************************************/
int getRoute (char *dnid, int lt_rte, char *nnid);
\end{lstlisting}




% Variable local para emacs, para  que encuentre el fichero maestro de
% compilaci�n y funcionen mejor algunas teclas r�pidas de AucTeX
%%%
%%% Local Variables:
%%% mode: latex
%%% TeX-master: "../Tesis.tex"
%%% End:
