%---------------------------------------------------------------------
%
%                          Cap�tulo 9
%
%---------------------------------------------------------------------

\chapter{Herramientas Software}
\label{cap9}
%\begin{FraseCelebre}
%\begin{Frase}
%At last the long wait is over \\
%the weight is off my shoulders \\
%I'm taking all control, yeah
%\end{Frase}
%\begin{Fuente}
%Thomas \& Guy-Manuel, Daft Punk
%\end{Fuente}
%\end{FraseCelebre}

\begin{resumen}
En este cap�tulo de ver�n las herramientas software empleadas, como pueden ser las de programaci�n del micro-controlador o las que se han empleado para la creaci�n de piezas 3D.
\end{resumen}




%-------------------------------------------------------------------
\section{Autodesk Inventor}
%-------------------------------------------------------------------
\label{cap9:sec:inventor}
%-------------------------------------------------------------------

Este software ha sido el empleado para el dise�o de las piezas 3D que conforman el generador. 
\medskip

Es un software de la compa��a Autodesk de modelado par�metrico. 
\medskip

Realizando operaciones booleanas sobre formas b�sicas (cubos, cilindros y esferas) se puede construir cualquier figura b�sica. 

\begin{figure}[!h]
        \centering
		\includegraphics[width=0.7\textwidth]{Imagenes/imagenes/cap9/inventor}  
        \caption{Software Inventor}
        \label{fig:cap9:inventor}
\end{figure}
\FloatBarrier

Otros softwares para este cometido son Blender, Open Scad, Solid Work o Solid edge.
\medskip

Lo importante tras generar el modelo 3D, es exportarlo en formato .stl.
%-------------------------------------------------------------------
\section{Cura}
%-------------------------------------------------------------------
\label{cap9:sec:cura}
%-------------------------------------------------------------------

Tras haber generado un archivo .stl con el inventor o alg�n otro programa de dise�o 3D, es necesario acondicionarlo para que pueda ser impreso.
\medskip

El funcionamiento de una impresora 3D se basa en impresi�n de capas horizontales. Es decir, no imprime la figura de una vez, sino que imprime capa a capa.
\medskip

Cada capa a su vez se compone de GCodes, que son ordenes de movimiento simples que un \ac{MCU} se encarga de transformar en pasos de un motor bipolar.
\medskip

Por esto es necesario un software que descomponga capa a capa la pieza 3D que hay en el archivo .stl y que adem�s las traduzca a GCode.
\medskip

Esta funci�n la realizan los software slicer (loncheador). 
\medskip

Un ejemplo es el software Cura \cite{cura}.

\begin{figure}[!h]
        \centering
		\includegraphics[width=0.7\textwidth]{Imagenes/imagenes/cap9/cura}  
        \caption{Software Cura}
        \label{fig:cap9:cura}
\end{figure}
\FloatBarrier


%-------------------------------------------------------------------
\section{MPLAB X}
%-------------------------------------------------------------------
\label{cap9:sec:mplab}
%-------------------------------------------------------------------
MPLAB X \ac{IDE} es un software multi-plataforma para desarrollar aplicaciones para los \ac{MCU} de Microchip.
\medskip

Se llama entorno de desarrollo integrado porque proporciona una plataforma unica para programar, desde el compilador hasta el cargador del programa en el \ac{MCU}. Esta basado en NetBeans de Oracle.

Algunas de sus caracter�sticas son:

\begin{itemize}
\item Soporte de m�ltiples configuraciones para cada proyecto.
\item Herramientas de depurado.
\item Proporciona enlaces r�pidos a las declaraciones de funciones o variables.
\end{itemize}

\begin{figure}[!h]
        \centering
		\includegraphics[width=0.9\textwidth]{Imagenes/imagenes/cap9/mplab}  
        \caption{MPLAB X con un programa de PIC 16}
        \label{fig:cap9:mpalb}
\end{figure}
\FloatBarrier

%-------------------------------------------------------------------
\subsection{Programador: ICD 3}
%-------------------------------------------------------------------
\label{cap9:sec:icd3}
%-------------------------------------------------------------------

Este dispositivo, es el encargado de cargar el programa en el \ac{MCU}. Funciona en conjunto con MPLAB X.

Ademas proporciona la capacidad de depurar (debug) usando el \ac{IDE}.
\medskip

Su interfaz con el ordenador es mediante un bus \ac{USB}, y con el \ac{MCU} mediante un conector RJ11.

\begin{figure}[!h]
        \centering
		\includegraphics[width=0.7\textwidth]{Imagenes/imagenes/cap9/ICD3}  
        \caption{Interfaz del IDC 3 con el MCU}
        \label{fig:cap9:icd3}
\end{figure}
\FloatBarrier







%
% Variable local para emacs, para  que encuentre el fichero maestro de
% compilaci�n y funcionen mejor algunas teclas r�pidas de AucTeX
%%%
%%% Local Variables:
%%% mode: latex
%%% TeX-master: "../Tesis.tex"
%%% End:
