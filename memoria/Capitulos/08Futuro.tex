%---------------------------------------------------------------------
%
%                          Cap�tulo 8
%
%---------------------------------------------------------------------

\chapter{Lineas Futuras}
\label{cap8}
%\begin{FraseCelebre}
%\begin{Frase}
%Like the legend of the phoenix\\
%all ends with beginnings\\
%what keeps the planet spinning\\
%the force from the beginning\\
%\end{Frase}
%\begin{Fuente}
%Thomas \& Guy-Manuel, Daft Punk
%\end{Fuente}
%\end{FraseCelebre}

\begin{resumen}
En este capitulo se expondr�n las lineas futuras del proyecto, mejoras y lineas de trabajo en las que, con investigaci�n, se pueden obtener mejores resultados.
\end{resumen}

\section{Introducci�n}


Puesto que este proyecto es una plataforma para la investigaci�n en energy harvesting con generadores inerciales, se pueden proponer muchas lineas de investigaci�n pero sobre todo mejoras espec�ficas sobre este generador para dejar un entorno funcional y estable.


%-------------------------------------------------------------------
\section{Posibles mejoras}
%-------------------------------------------------------------------
\label{cap8:sec:mejoras}
%-------------------------------------------------------------------


Las mejoras son modificaciones sobre el proyecto actual, que mejorar�an su rendimiento.

\subsection{Nuevos dise�os de piezas}
Las piezas dise�adas hasta ahora (Base,Sintonizador,etc) podr�an ser optimizadas, buscando reducci�n de peso o mejora de los puntos de equilibrio.
\subsection{Detecci�n de limite de recorrido}
El generador actual no limita el recorrido, no se puede detectar cuando el sintonizador llega al final del su recorrido. Por lo tanto en el estado actual, el motor seguir�a girando aun cuando el sintonizador no puede moverse mas. Esto causar�a que alguna parte se rompiese.
\bigskip

Por esto, como mejora se propone la inclusi�n de sensores fin de carrera que sean activados por el sintonizador cuando se acerque al limite de su recorrido. Una vez activados, el \ac{MCU} actualizar�a en su memoria la posici�n del sintonizador y actuaria en consecuencia (No permitiendo movimientos mayores de donde se encuentra en ese momento)


%-------------------------------------------------------------------
\section{Estudios futuros}
%-------------------------------------------------------------------
\label{cap8:sec:furtherStudies}
%-------------------------------------------------------------------

Las lineas futuras de este proyecto comprenden las modificaciones a las que se le puede someter y que implicar�an un cambio en el concepto.

\subsection{Reducci�n de tama�o}

Cuando se hayan realizado todas las pruebas sobre la plataforma actual, se puede investigar en la reducci�n del tama�o del generador. Las piezas imprimibles, no supondr�an problema, pero no as� la l�mina y la bobina que requerir�an otro estudio

\subsection{Auto-Alimentaci�n}

\subsection{B�squeda de mejores materiales}

% Variable local para emacs, para  que encuentre el fichero maestro de
% compilaci�n y funcionen mejor algunas teclas r�pidas de AucTeX
%%%
%%% Local Variables:
%%% mode: latex
%%% TeX-master: "../Tesis.tex"
%%% End:
