%---------------------------------------------------------------------
%
%                          Cap�tulo 1
%
%---------------------------------------------------------------------

\chapter{Introducci�n}

\begin{FraseCelebre}
\begin{Frase}
If you hear any noise,\\
hit me
\end{Frase}
\begin{Fuente}
Thomas \& Guy-Manuel, Daft Punk
\end{Fuente}
\end{FraseCelebre}

\begin{resumen}
In this chapter, a global description of the project is given. This global description covers an introduction to the context as well as the need for, and a brief description of the work. It defines goals and the project organization over time. In addition, the structure of this dissertation is described.  
\end{resumen}


%-------------------------------------------------------------------
\section{Background}
%-------------------------------------------------------------------
\label{cap1:sec:background}
%-------------------------------------------------------------------

 A \ac{WSN} consists of a spatially distributed number of sensors spread across a geographical 

%-------------------------------------------------------------------
\section{Goals}
%-------------------------------------------------------------------
\label{cap1:sec:goals}
%-------------------------------------------------------------------


%-------------------------------------------------------------------
\section{Outline}
%-------------------------------------------------------------------
\label{cap1:sec:outline}
%-------------------------------------------------------------------







% Variable local para emacs, para  que encuentre el fichero maestro de
% compilaci�n y funcionen mejor algunas teclas r�pidas de AucTeX
%%%
%%% Local Variables:
%%% mode: latex
%%% TeX-master: "../Tesis.tex"
%%% End:
