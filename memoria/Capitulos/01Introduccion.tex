%---------------------------------------------------------------------
%
%                          Cap�tulo 1
%
%---------------------------------------------------------------------

\chapter{Introducci�n}

%\begin{FraseCelebre}
%\begin{Frase}
%If you hear any noise,\\
%hit me
%\end{Frase}
%\begin{Fuente}
%Thomas \& Guy-Manuel, Daft Punk
%\end{Fuente}
%\end{FraseCelebre}

\begin{resumen}
En este capitulo se da una descripci�n global del proyecto. Esta descripci�n abarca tanto el contexto como las necesidades que cubre el proyecto. Se describen los objetivos y la organizaci�n del trabajo. 
\end{resumen}



%-------------------------------------------------------------------
\section{Nubes de sensores inal�mbricos}
%-------------------------------------------------------------------
\label{cap1:sec:sensores}
%-------------------------------------------------------------------


%-------------------------------------------------------------------
\section{Problemas de alimentaci�n y Energy Harvesting}
%-------------------------------------------------------------------
\label{cap1:sec:alim}
%-------------------------------------------------------------------

Las opciones actuales para alimentar un nodo inal�mbrico son dos. 
\smallskip
Se pueden emplear bater�as con su consiguiente ciclo de agotamiento y recarga o, m�s a largo plazo, su sustituci�n por otra, o se pueden emplear m�todos de recolecci�n de energ�a.


El termino ?Energy Harvesting? o recolecci�n de energ�a cubre el campo de la conversi�n de energ�a disponible en el entorno en energ�a el�ctrica. El principal inter�s en este campo deriva de su capacidad de actuar como un suministro de energ�a independiente para sistemas inal�mbricos auto alimentados, en contraste con el uso de bater�as.
El ?energy harvesting? basado en vibraciones que aprovecha la energ�a cin�tica presente en el entorno es solo uno de los tipos que existen.
Tambi�n se puede producir energ�a el�ctrica mediante efectos fotovoltaicos o termoel�ctricos, en funci�n de la luz o la temperatura del entorno. 
\bigskip

La generaci�n a partir de vibraciones es especialmente adecuada para entornos industriales donde los movimientos y vibraciones son frecuentes, y los sensores pueden estar colocados en lugares oscuros y sucios.
\medskip

Otra fuente de vibraciones para ?energy harvesting? es el cuerpo humano, donde un amplio rango de dispositivos ha sido desarrollado para extraer su energ�a de las pisadas (deceleraci�n brusca), movimiento de mu�eca (relojes cin�ticos) o sistemas montados en distintas prendas y complementos donde se den esas aceleraciones necesarias.
\medskip

En aplicaciones basadas en m�quinas, los niveles de vibraci�n pueden ser muy peque�os ($<1ms^{-2}$) a frecuencias que normalmente coinciden con la frecuencia del suministro de energ�a de la maquina (50 Hz). Estas amplitudes tan peque�as fuerzan que la manera mas eficiente de extraer energ�a de estas vibraciones sea mediante generadores inerciales con una frecuencia de resonancia caracter�stica.



%-------------------------------------------------------------------
\section{Generadores electromagn�ticos}
%-------------------------------------------------------------------
\label{cap1:sec:magnet}
%-------------------------------------------------------------------







% Variable local para emacs, para  que encuentre el fichero maestro de
% compilaci�n y funcionen mejor algunas teclas r�pidas de AucTeX
%%%
%%% Local Variables:
%%% mode: latex
%%% TeX-master: "../Tesis.tex"
%%% End:
