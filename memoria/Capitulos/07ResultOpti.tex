%---------------------------------------------------------------------
%
%                          Cap�tulo 7
%
%---------------------------------------------------------------------

\chapter{Resultados y optimizaci�n}
\label{cap7}
%\begin{FraseCelebre}
%\begin{Frase}
%At last the long wait is over \\
%the weight is off my shoulders \\
%I'm taking all control, yeah
%\end{Frase}
%\begin{Fuente}
%Thomas \& Guy-Manuel, Daft Punk
%\end{Fuente}
%\end{FraseCelebre}

\begin{resumen}
En este cap�tulo se incluyen los resultados obtenidos tras las pruebas realizadas al prototipo y loas mejoras realizadas sobre el prototipo b�sico.
\end{resumen}

%-------------------------------------------------------------------
\section{Resultados}
%-------------------------------------------------------------------
\label{cap7:sec:mplabx}
%-------------------------------------------------------------------

Se van a realizar pruebas por m�dulos, es decir se probar�n las distintas partes del proyecto por separado y finalmente si todas las pruebas son satisfactorias se realizar� la prueba final integrando todo el proyecto.

\subsection{Amplificador de se�al}
Este amplificador visto en \ref{cap5:subsec:ampli} sirve para adecuar la se�al que entrega el generador de funciones, al generador de vibraciones.

\subsubsection{Prueba de funcionamiento}
Como ya se vio en \ref{cap5:subsec:ampli}, el circuito amplifica y mantiene la frecuencia de la se�al original. Por lo tanto pasa la prueba b�sica de funcionamiento.

\subsubsection{Prueba de resistencia}

Durante la primera prueba se vio que los transistores alcanzaban una temperatura alta, y por lo tanto el disipador tambi�n, por esto se decidi� realizar una prueba de resistencia, es decir, dejar el amplificador funcionando para ver si en alg�n momento esta alta temperatura causaba alg�n problema.

\begin{itemize}
\item Se pone en funcioanmiento el amplificador con el generador de vibraciones a la salida.
\item Tras 30 minutos en funcionamiento, el amplificador sigue funcionando con normalidad aunque el disipador alcance una alta temperatura.
\item No se aprecian distorsiones en la salida, por lo tanto pasa la prueba de resistencia.
\end{itemize}
 

%-------------------------------------------------------------------
\section{Optimizaci�n}
%-------------------------------------------------------------------
\label{cap7:sec:icd3}
%-------------------------------------------------------------------





%
% Variable local para emacs, para  que encuentre el fichero maestro de
% compilaci�n y funcionen mejor algunas teclas r�pidas de AucTeX
%%%
%%% Local Variables:
%%% mode: latex
%%% TeX-master: "../Tesis.tex"
%%% End:
