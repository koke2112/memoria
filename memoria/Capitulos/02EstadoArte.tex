%---------------------------------------------------------------------
%
%                          Cap�tulo 2
%
%---------------------------------------------------------------------

\chapter{Micro Generadores de ``Energy Harvesting'': Estado del arte}

%\begin{FraseCelebre}
%\begin{Frase}
%Television rules the nation
%\end{Frase}
%\begin{Fuente}
%Thomas \& Guy-Manuel, Daft Punk
%\end{Fuente}
%\end{FraseCelebre}

\begin{resumen}
A la hora de realizar cualquier proyecto es necesario realizar previamente un estudio del campo en el cual se va a trabajar para conocer sus puntos fuertes y d�biles, saber donde se investiga y el estado actual para conocer la base sobre la que se va a empezar a trabajar.
\end{resumen}


%-------------------------------------------------------------------
\section{Introducci�n}
%-------------------------------------------------------------------
\label{cap2:sec:cognitiveRadio}
%-------------------------------------------------------------------

Del estudio del estado del campo de energy harvesting con generadores inerciales se obtienen resultados un tanto dispares, por un lado la simpleza de este tipo de generadores hace que la mayor�a de los estudios disponibles carezcan de originalidad puesto que poco nuevo se puede investigar en un sistema conocido y estudiado desde hace mucho tiempo como es el de una lamina que vibra.
Por otro lado, est�n empezando a aparecer estudios que se centran en optimizar la eficiencia de estos generadores, ya sea amplificando su ancho de banda u optimizando el flujo magn�tico que atraviesa la bobina en funci�n de los imanes, las posibles arquitecturas de construcci�n como puede ser la diferentes formas de colocar la bobina respecto al im�n o incluso las distintas formas de construir una bobina.
Otras de las conclusiones que se han obtenido es la idea de frecuencia de resonancia. Para las laminas vibrantes, y en funci�n de sus caracter�sticas f�sicas, existe una frecuencia a la cual la lamina transfiere toda la potencia de las vibraciones de su entorno. Esto se puede utilizar para, trabajando a esta potencia, transmitir la m�xima potencia a un transductor y as� transformar la m�xima energ�a de vibraci�n posible en energ�a el�ctrica.

Dada la sencillez de este tipo de generadores, ya existen soluciones comerciales, aunque est�n poco depuradas en lo que a eficiencia se refiere siendo el principal problema de los generadores basados en vibraciones, la selecci�n de la frecuencia de resonancia.
Por otro lado hay ciertos grupos de investigaci�n en universidades como las que forman el consorcio EPSRC, universidad de southhampton, de newcastle, de bristol y el colegio imperial de londres \cite{epsrc}  que se dedican a desarrollar sistemas y estudios sobre energy harvesting.


%-------------------------------------------------------------------
\section{Soluciones comerciales}
%-------------------------------------------------------------------
\label{cap2:sec:cognitiveNetworks}
%-------------------------------------------------------------------
Actualmente existen soluciones comerciales de estos generadores
http://www.microstrain.com/energy-harvesting/harvesters
Microstrain produce dos tipos, electromagneticos y piezoelectricos. Estos generadores funcionan en el rango de 15-60Hz y su frecuencia de resonancia se puede fijar de forma manual antes del funcionamiento.
http://www.microgensystems.co/products.asp
Esta compa��a produce unos microgeneradores de frecuencia fija, por lo que previamente a adquirirlo habr�a que caracterizar el entorno vibrante y encontrar la frecuencia a la que las vibraciones tienen mayor potencia.
http://www.innowattech.co.il/index.aspx (piezoelectrico)
Innowattech proporciona soluciones basadas en microgeneradores piezoel�ctricos.
http://www.perpetuum.com/
Esta empresa que nacio como un spin-off de la universidad de southhampton se dedica a la comercializacion de generadores electromagneticos, cuya frecuencia de resonancia se puede fijar manualmente.      

 
%-------------------------------------------------------------------
\section{Wireless Sensor Networks}
%-------------------------------------------------------------------
\label{cap2:sec:wirelessSensorNetworks}
%-------------------------------------------------------------------



%-------------------------------------------------------------------
\section{Cognitive Wireless Sensor Networks}
%-------------------------------------------------------------------
\label{cap2:sec:cognitiveWirelessSensorNetworks} 
%-------------------------------------------------------------------


 
%-------------------------------------------------------------------
\subsection{Current Implementations}
%-------------------------------------------------------------------
\label{cap2:sec:currentImplementations}
%-------------------------------------------------------------------


%-------------------------------------------------------------------
\subsubsection{Software platforms - simulators}
%-------------------------------------------------------------------



%-------------------------------------------------------------------
\subsubsection{Hardware platforms}
%-------------------------------------------------------------------

%-------------------------------------------------------------------
\subsubsection{Standards}
%-------------------------------------------------------------------


%-------------------------------------------------------------------
\section{Contribution}
%-------------------------------------------------------------------
\label{cap2:sec:cotribution}
%-------------------------------------------------------------------



%-------------------------------------------------------------------
%\section*{\NotasBibliograficas}
%-------------------------------------------------------------------
%\TocNotasBibliograficas

%Citamos algo para que aparezca en la bibliograf�a\ldots
%\citep{ldesc2e}

%\medskip

%Y tambi�n ponemos el acr�nimo \ac{CVS} para que no cruja.

%Ten en cuenta que si no quieres acr�nimos (o no quieres que te falle la compilaci�n en ``release'' mientras no tengas ninguno) basta con que no definas la constante \verb+\acronimosEnRelease+ (en \texttt{config.tex}).


%-------------------------------------------------------------------
%\section*{\ProximoCapitulo}
%-------------------------------------------------------------------
%\TocProximoCapitulo

%...

% Variable local para emacs, para  que encuentre el fichero maestro de
% compilaci�n y funcionen mejor algunas teclas r�pidas de AucTeX
%%%
%%% Local Variables:
%%% mode: latex
%%% TeX-master: "../Tesis.tex"
%%% End:
